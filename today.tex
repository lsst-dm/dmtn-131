\section{Studies to date}
A simplified sizing model (\citeds{DMTN-072}) was used to run a Google study on \gls{LSST}, a detailed
report may be found in \citeds{DMTN-125}.

We deomstrated some of the major compnents of \gls{LSST} Data Managmenet could work on Google.
Brifely we deployed \gls{Qserv} on Google with reasonable performance (80\% or better of in-house).
We demonstrated Data transfer  could adequate for \gls{Prompt Processing}, within the limits of the available network. i
The Prompt Product Database was  stood up and tested.  The  \gls{Science Platform} was deployed and users simulated.
The later of course is designed around kubernetes and almost made for Google Cloud.


We did not demonstrate teh other major (and largest part of \gls{DM}) the \gls{Data Release} Processing. That we are
currently trying with
Amazon Web Services / Elastic Compute Cloud and HTCondor.
See \citeds{DMTN-114} for details.
This is progressing.

We ay conclude from these studies that we have good people in \gls{DM} and
we are able to deploy our systems in various locations especially if kubernetes is avialble.
