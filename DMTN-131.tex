\documentclass[DM,TN,lsstdraft]{lsstdoc}
% lsstdoc documentation: https://lsst-texmf.lsst.io/lsstdoc.html

\input{meta}

% Package imports go here.

% Local commands go here.
% DO NOT EDIT - generated by /Users/womullan/LSSTgit/lsst-texmf/bin/generateAcronyms.py from https://lsst-texmf.lsst.io/.
\newglossaryentry{AURA} {name={AURA}, description={\gls{Association of Universities for Research in Astronomy}}}
\newglossaryentry{Alert} {name={Alert}, description={A packet of information for each source detected with signal-to-noise ratio > 5 in a difference image during Prompt Processing, containing measurement and characterization parameters based on the past 12 months of LSST observations plus small cutouts of the single-visit, template, and difference images, distributed via the internet}}
\newglossaryentry{Alert Production} {name={Alert Production}, description={The principal component of Prompt Processing that processes and calibrates incoming images, performs Difference Image Analysis to identify DIASources and DIAObjects, packages and distributes the resulting Alerts, and runs the Moving Object Processing System}}
\newglossaryentry{Archive} {name={Archive}, description={The repository for documents required by the NSF to be kept. These include documents related to design and development, construction, integration, test, and operations of the LSST observatory system. The archive is maintained using the enterprise content management system DocuShare, which is accessible through a link on the project website www.project.lsst.org}}
\newglossaryentry{Archive Center} {name={Archive Center}, description={Part of the LSST Data Management System, the LSST archive center is a data center at NCSA that hosts the LSST Archive, which includes released science data and metadata, observatory and engineering data, and supporting software such as the LSST Software Stack}}
\newglossaryentry{Association of Universities for Research in Astronomy} {name={Association of Universities for Research in Astronomy}, description={ consortium of US institutions and international affiliates that operates world-class astronomical observatories, AURA is the legal entity responsible for managing what it calls independent operating Centers, including LSST, under respective cooperative agreements with the National Science Foundation. AURA assumes fiducial responsibility for the funds provided through those cooperative agreements. AURA also is the legal owner of the AURA Observatory properties in Chile}}
\newglossaryentry{Center} {name={Center}, description={An entity managed by AURA that is responsible for execution of a federally funded project}}
\newacronym{DCR} {DCR} {Differential Chromatic Refraction}
\newacronym{DIA} {DIA} {Difference Image Analysis}
\newglossaryentry{DIAObject} {name={DIAObject}, description={A DIAObject is the association of DIASources, by coordinate, that have been detected with signal-to-noise ratio greater than 5 in at least one difference image. It is distinguished from a regular Object in that its brightness varies in time, and from a SSObject in that it is stationary (non-moving)}}
\newglossaryentry{DIASource} {name={DIASource}, description={A DIASource is a detection with signal-to-noise ratio greater than 5 in a difference image}}
\newacronym{DM} {DM} {\gls{Data Management}}
\newacronym{DMS} {DMS} {Data Management Subsystem}
\newglossaryentry{DMTN} {name={DMTN}, description={DM Technical Note}}
\newacronym{DOE} {DOE} {\gls{Department of Energy}}
\newacronym{DR} {DR} {Data Release}
\newacronym{DRP} {DRP} {Data Release Production}
\newglossaryentry{Data Management} {name={Data Management}, description={The LSST Subsystem responsible for the Data Management System (DMS), which will capture, store, catalog, and serve the LSST dataset to the scientific community and public. The DM team is responsible for the DMS architecture, applications, middleware, infrastructure, algorithms, and Observatory Network Design. DM is a distributed team working at LSST and partner institutions, with the DM Subsystem Manager located at LSST headquarters in Tucson}}
\newglossaryentry{Data Management Subsystem} {name={Data Management Subsystem}, description={The subsystems within Data Management may contain a defined combination of hardware, a software stack, a set of running processes, and the people who manage them: they are a major component of the DM System operations. Examples include the 'Archive Operations Subsystem' and the 'Data Processing Subsystem'"."}}
\newglossaryentry{Data Management System} {name={Data Management System}, description={The computing infrastructure, middleware, and applications that process, store, and enable information extraction from the LSST dataset; the DMS will process peta-scale data volume, convert raw images into a faithful representation of the universe, and archive the results in a useful form. The infrastructure layer consists of the computing, storage, networking hardware, and system software. The middleware layer handles distributed processing, data access, user interface, and system operations services. The applications layer includes the data pipelines and the science data archives' products and services}}
\newglossaryentry{Data Release} {name={Data Release}, description={The approximately annual reprocessing of all LSST data, and the installation of the resulting data products in the LSST Data Access Centers, which marks the start of the two-year proprietary period}}
\newglossaryentry{Data Release Processing} {name={Data Release Processing}, description={Deprecated term; see Data Release Production}}
\newglossaryentry{Data Release Production} {name={Data Release Production}, description={An episode of (re)processing all of the accumulated LSST images, during which all output DR data products are generated. These episodes are planned to occur annually during the LSST survey, and the processing will be executed at the Archive Center. This includes Difference Imaging Analysis, generating deep Coadd Images, Source detection and association, creating Object and Solar System Object catalogs, and related metadata}}
\newglossaryentry{Department of Energy} {name={Department of Energy}, description={cabinet department of the United States federal government; the DOE has assumed technical and financial responsibility for providing the LSST camera. The DOE's responsibilities are executed by a collaboration led by SLAC National Accelerator Laboratory}}
\newglossaryentry{Difference Image} {name={Difference Image}, description={Refers to the result formed from the pixel-by-pixel difference of two images of the sky, after warping to the same pixel grid, scaling to the same photometric response, matching to the same PSF shape, and applying a correction for Differential Chromatic Refraction. The pixels in a difference thus formed should be zero (apart from noise) except for sources that are new, or have changed in brightness or position. In the LSST context, the difference is generally taken between a visit image and template. }}
\newglossaryentry{Difference Image Analysis} {name={Difference Image Analysis}, description={The detection and characterization of sources in the Difference Image that are above a configurable threshold, done as part of Alert Generation Pipeline}}
\newglossaryentry{Differential Chromatic Refraction} {name={Differential Chromatic Refraction}, description={The refraction of incident light by Earth's atmosphere causes the apparent position of objects to be shifted, and the size of this shift depends on both the wavelength of the source and its airmass at the time of observation. DCR corrections are done as a part of DIA}}
\newglossaryentry{DocuShare} {name={DocuShare}, description={The trade name for the enterprise management software used by LSST to archive and manage documents}}
\newacronym{FITS} {FITS} {\gls{Flexible Image Transport System}}
\newglossaryentry{Flexible Image Transport System} {name={Flexible Image Transport System}, description={an international standard in astronomy for storing images, tables, and metadata in disk files. See the IAU FITS Standard for details}}
\newacronym{IAU} {IAU} {International Astronomical Union}
\newacronym{LSST} {LSST} {Large Synoptic Survey Telescope}
\newacronym{MOPS} {MOPS} {Moving Object Processing System}
\newglossaryentry{Moving Object Processing System} {name={Moving Object Processing System}, description={The Moving Object Processing System (MOPS) identifies new SSObjects using unassociated DIASources. MOPS is part of the Science Pipelines}}
\newglossaryentry{NCSA} {name={NCSA}, description={National Center for Supercomputing Applications}}
\newacronym{NSF} {NSF} {\gls{National Science Foundation}}
\newglossaryentry{National Science Foundation} {name={National Science Foundation}, description={primary federal agency supporting research in all fields of fundamental science and engineering; NSF selects and funds projects through competitive, merit-based review}}
\newglossaryentry{Object} {name={Object}, description={In LSST nomenclature this refers to an astronomical object, such as a star, galaxy, or other physical entity. E.g., comets, asteroids are also Objects but typically called a Moving Object or a Solar System Object (SSObject). One of the DRP data products is a table of Objects detected by LSST which can be static, or change brightness or position with time}}
\newglossaryentry{Operations} {name={Operations}, description={The 10-year period following construction and commissioning during which the LSST Observatory conducts its survey}}
\newacronym{PSF} {PSF} {Point Spread Function}
\newglossaryentry{Project Manager} {name={Project Manager}, description={The person responsible for exercising leadership and oversight over the entire LSST project; he or she controls schedule, budget, and all contingency funds}}
\newglossaryentry{Prompt Processing} {name={Prompt Processing}, description={The processing that occurs at the Archive Center on the nightly stream of raw images coming from the telescope, including Difference Imaging Analysis, Alert Production, and the Moving Object Processing System. This processing generates Prompt Data Products}}
\newglossaryentry{Qserv} {name={Qserv}, description={Proprietary Database built by SLAC for LSST}}
\newglossaryentry{SLAC} {name={SLAC}, description={No longer an acronym; formerly Stanford Linear Accelerator Center}}
\newglossaryentry{Science Pipelines} {name={Science Pipelines}, description={The library of software components and the algorithms and processing pipelines assembled from them that are being developed by DM to generate science-ready data products from LSST images. The Pipelines may be executed at scale as part of LSST Prompt or Data Release processing, or pieces of them may be used in a standalone mode or executed through the LSST Science Platform. The Science Pipelines are one component of the LSST Software Stack}}
\newglossaryentry{Science Platform} {name={Science Platform}, description={A set of integrated web applications and services deployed at the LSST Data Access Centers (DACs) through which the scientific community will access, visualize, and perform next-to-the-data analysis of the LSST data products}}
\newglossaryentry{Software Stack} {name={Software Stack}, description={Often referred to as the LSST Stack, or just The Stack, it is the collection of software written by the LSST Data Management Team to process, generate, and serve LSST images, transient alerts, and catalogs. The Stack includes the LSST Science Pipelines, as well as packages upon which the DM software depends. It is open source and publicly available}}
\newglossaryentry{Solar System Object} {name={Solar System Object}, description={A solar system object is an astrophysical object that is identified as part of the Solar System: planets and their satellites, asteroids, comets, etc. This class of object had historically been referred to within the LSST Project as Moving Objects}}
\newglossaryentry{Source} {name={Source}, description={A single detection of an astrophysical object in an image, the characteristics for which are stored in the Source Catalog of the DRP database. The association of Sources that are non-moving lead to Objects; the association of moving Sources leads to Solar System Objects. (Note that in non-LSST usage "source" is often used for what LSST calls an Object.)}}
\newglossaryentry{Subsystem} {name={Subsystem}, description={A set of elements comprising a system within the larger LSST system that is responsible for a key technical deliverable of the project}}
\newglossaryentry{Subsystem Manager} {name={Subsystem Manager}, description={responsible manager for an LSST subsystem; he or she exercises authority, within prescribed limits and under scrutiny of the Project Manager, over the relevant subsystem's cost, schedule, and work plans}}
\newacronym{US} {US} {United States}
\newglossaryentry{airmass} {name={airmass}, description={The pathlength of light from an astrophysical source through the Earth's atmosphere. It is given approximately by sec z, where z is the angular distance from the zenith (the point directly overhead, where airmass = 1.0) to the source}}
\newglossaryentry{astronomical object} {name={astronomical object}, description={A star, galaxy, asteroid, or other physical object of astronomical interest. Beware: in non-LSST usage, these are often known as sources}}
\newglossaryentry{camera} {name={camera}, description={An imaging device mounted at a telescope focal plane, composed of optics, a shutter, a set of filters, and one or more sensors arranged in a focal plane array}}
\newglossaryentry{flux} {name={flux}, description={Shorthand for radiative flux, it is a measure of the transport of radiant energy per unit area per unit time. In astronomy this is usually expressed in cgs units: erg/cm2/s}}
\newglossaryentry{metadata} {name={metadata}, description={General term for data about data, e.g., attributes of astronomical objects (e.g. images, sources, astroObjects, etc.) that are characteristics of the objects themselves, and facilitate the organization, preservation, and query of data sets. (E.g., a FITS header contains metadata)}}
\newglossaryentry{shape} {name={shape}, description={In reference to a Source or Object, the shape is a functional characterization of its spatial intensity distribution, and the integral of the shape is the flux. Shape characterizations are a data product in the DIASource, DIAObject, Source, and Object catalogs}}
\newglossaryentry{stack} {name={stack}, description={a grouping, usually in layers (hence stack), of software packages and services to achieve a common goal. Often providing a higher level set of end user oriented services and tools}}
\newglossaryentry{transient} {name={transient}, description={A transient source is one that has been detected on a difference image, but has not been associated with either an astronomical object or a solar system body}}

\makeglossaries


\title{When  clouds might be good  for LSST}

% Optional subtitle
% \setDocSubtitle{A subtitle}

\author{%
William O'Mullane
}

\setDocRef{DMTN-131}
\setDocUpstreamLocation{\url{https://github.com/lsst-dm/dmtn-131}}

\date{\vcsDate}

% Optional: name of the document's curator
% \setDocCurator{The Curator of this Document}

\setDocAbstract{%
In this short note we would like to consider potential annual operating costs for LSST as well as discuss long term archiving. The goal would be to see if we can come to an agreement with a major cloud provider.
}

% Change history defined here.
% Order: oldest first.
% Fields: VERSION, DATE, DESCRIPTION, OWNER NAME.
% See LPM-51 for version number policy.
\setDocChangeRecord{%
  \addtohist{1}{YYYY-MM-DD}{Unreleased.}{William O'Mullane}
}

\setDocCompact{true}
\begin{document}

% Create the title page.
\mkshorttitle

% ADD CONTENT HERE
% You can also use the \input command to include several content files.
\section{Introduction}
The Large Synoptic Survey Telescope (\gls{LSST}) will usher in a new era of data-intensive astronomy. The observing program will observe the southern sky repeatedly over 10 years in 6 bands providing an unprecedented census of the astrophysical bodies in the universe.  Funded by the \gls{NSF} and the \gls{DOE}, this keystone observatory is due to go into operations in October 2022.

Producing 20TB of data a night, this is a huge step up in data acquisition from other optical telescopes. At its conception this was considered an ominous data volume requiring highly specialized computing infrastructure. In the intervening time, however, the growth of planetary-scale industry services (such us Google or Facebook) has resulted in software, engineering techniques and infrastructure that render this sort of data flow routine. \gls{LSST}  operations are expected to cost tens of millions a year with order of \$10M computing budget.

The \gls{LSST} computing load is a poster child for cloud computing - the science platform is designed for kubernetes and the
data release processing fits perfectly with opportunistic compute pricing. This is a large \gls{NSF} project and the the one most suited and ready for cloud deployment.
We have used some pathfinder deployments to demonstrate that it is feasible to use commercial cloud providers for the \gls{LSST} \gls{Data Management} production system. Such a move would bring significant technological and operational advantages; the barrier to acting on this is price and uncertainty on future pricing.

A solution might be to reach a fixed price partnership for a cloud-based deployment of the \gls{Data Management} systems in which Google undertake to provide \emph{do what is needed} for success at some reasonable and agreed-upon annual fee.

\section{Studies to date}
A simplified sizing model (\citeds{DMTN-072}) was used to run a Google study on \gls{LSST}, a detailed
report may be found in \citeds{DMTN-125}.

We deomstrated some of the major compnents of \gls{LSST} Data Managmenet could work on Google.
Brifely we deployed \gls{Qserv} on Google with reasonable performance (80\% or better of in-house).
We demonstrated Data transfer  could adequate for \gls{Prompt Processing}, within the limits of the available network. i
The Prompt Product Database was  stood up and tested.  The  \gls{Science Platform} was deployed and users simulated.
The later of course is designed around kubernetes and almost made for Google Cloud.


We did not demonstrate teh other major (and largest part of \gls{DM}) the \gls{Data Release} Processing. That we are
currently trying with
Amazon Web Services / Elastic Compute Cloud and HTCondor.
See \citeds{DMTN-114} for details.
This is progressing.

We ay conclude from these studies that we have good people in \gls{DM} and
we are able to deploy our systems in various locations especially if kubernetes is avialble.

\section{LSST compute and storage needs}

The greatest cost driver is storage - we accumulate about 50PB a year of data. All of this needs to be processed annually. Hence in year 10 we need to access about half an Exabyte of data. Not all of this will be regularly accessed, it is likely few of the raw images will be reprocessed by individual astronomers.

\tabref{tab:Inputs} gives a rough overview of compute and storage needs.

\tiny \begin{longtable} { |p{0.22\textwidth}  |r  |r  |r  |r  |r |} 
\caption{Various inputs for deriving costs \label{tab:Inputs}}\\ 
\hline 
\textbf{Year}&\textbf{2019}&\textbf{2020}&\textbf{2021}&\textbf{2022}&\textbf{2023} \\ \hline
{FLOPs Needed Total (no Alerts)}&{1.00E+19}&{9.48261E+19}&{1.00E+19}&{4.74131E+20}&{5.91525E+20} \\ \hline
{Time to Process days}&{252.0}&{252.0}&{252.0}&{252.0}&{252.0} \\ \hline
{Time to Process seconds}&{21772800.0}&{21772800.0}&{21772800.0}&{21772800.0}&{21772800.0} \\ \hline
{Instantaneous GFLOP/ s}&{4.59E+02}&{4355.255691}&{4.59E+02}&{21776.27846}&{27168.07327} \\ \hline
{Instantaneous GFLOP/ s (inc Alerts)}&{4.59E+02}&{30025.25569}&{2.61E+04}&{21776.27846}&{27168.07327} \\ \hline
{Disk Space TB}&{5000}&{10000}&{20000}&{50000}&{100000} \\ \hline
{I/ O for year TB}&{15000}&{30000}&{60000}&{150000}&{300000} \\ \hline
{Base numbers }&{GFLOP}&&&& \\ \hline
{LDM-138 DR1,2 Data Rel sheet row 1}&{426717500000}&{}&{97381399021}&& \\ \hline
{LDM-138 DR3 Data Rel sheet row 2}&{959090000000}&&&& \\ \hline
{LDM-138 Alert Instananeous}&{25670}&&&& \\ \hline
{Alert Total, assuming 275k visits/ year}&{177219625000}&&&& \\ \hline
\textbf{Total Yr1 (inc DAC)}&\textbf{474130555556}&&&& \\ \hline
\end{longtable} \normalsize




\section{Cost}
Currently the cloud computing cost models do  not align well with federal research computing plans. We believe this gap will narrow or disappear in five years or so, however in the meantime LSST and Google may miss an opportunity to move a major research project on to commodity cloud computing.

Using the information from the Google study where we ran some of our real processes, we have come to a price for running the \gls{Science Platform} and storing data on Google. For 15PB of storage and a modest K8S cluster to host the platform the projected 2022 cost is around \$3M of which more than $\frac{2}{3}$ are storage costs.

Though a good price this is not a sustainable price for \gls{LSST}, we can construct petascale storage we would own and use for 5 years for under \$200K a petabyte (the implied \emph{annual} price at google).  We require about 50 Petabytes a year for 10 years. The out year costs look prohibitive on the cloud.

The cost of compute is probably not an issue in comparison - we can use spot/interuptable instance pricing for \gls{DRP}.

Based on the proof of concept some prices were calculated in \tabref{tab:Google}.

\tiny \begin{longtable} { |p{0.22\textwidth}  |r  |r  |r |} 
\caption{Price estimates from google POC \label{tab:Google}}\\ 
\hline 
{Google Compute}&{per month}&{1 year price}&{Price GFLOP} \\ \hline
{POC price GFLOP (n1-highmen-4)}&{\$242.00}&{\$2,904.00}&{\$2.60} \\ \hline
{LIkely (inefficeny included)}&{}&{}&{\$5.34} \\ \hline
{Pessimistic (double that)}&{}&{}&{\$10.69} \\ \hline
{Google Storage}&{GB/ m}&{TB/ month}&{note } \\ \hline
{Optm. }&{0.007}&{\$84.00}& \\ \hline
{Likely (HDD)}&{0.03999975641}&{\$480.00}&{26cent on web} \\ \hline
{Pessimiistic (SSD)}&{0.1699908088}&{\$2,039.89}& \\ \hline
\end{longtable} \normalsize




\section{Other issues to consider}
We may still wish to keep a copy of data out of the cloud e.g. in Chile and/or NCSA.

We have not considered bulk transfer of data to other partners - this may be far too expensive via google.

~



\appendix
% Include all the relevant bib files.
% https://lsst-texmf.lsst.io/lsstdoc.html#bibliographies
\label{sec:bib}
\bibliography{local,lsst,lsst-dm,refs_ads,refs,books}

% Make sure lsst-texmf/bin/generateAcronyms.py is in your path
%\section{Acronyms} \label{sec:acronyms}
%\input{acronyms.tex}
\printglossaries


\end{document}
