\section{Introduction}
The Large Synoptic Survey Telescope (\gls{LSST}) will usher in a new era of astronomy, the observing program will observe the southern sky repeatedly over 10 years in 6 bands providing an unprecedented  census of the astrophysical bodies in the universe.  Funded bu \gls{NSF} and \gls{DOE} this key observatory is due to go into operations in October 2022.

Producing 20TB a night this is a huge step up in data acquisition from other optical telescopes. At its conception it was considered an ominous data volume requiring highly specialized computing infrastructure. These days, however, industry is handling this sort of data flow without problems. LSST  operations are expected to cost tens of millions a year with order of \$10M computing budget.

We have made some experiments to show that we could potentially use commercial cloud providers for at least a large part of the \gls{LSST} \gls{Data Management}.  The barrier to acting on this is price and uncertainty on future pricing.

A solution might be to agree a fixed price partnership on the experiment where google agree to \emph{do what is needed} for success at some reasonable and agreed annual fee.

\section{Studies to date}
A simplified sizing model (\citeds{DMTN-072}) was used to run a Google study on \gls{LSST}, a detailed
report may be found in \citeds{DMTN-125}.

We deomstrated some of the major compnents of \gls{LSST} Data Managmenet could work on Google.
Brifely we deployed \gls{Qserv} on Google with reasonable performance (80\% or better of in-house).
We demonstrated Data transfer  could adequate for \gls{Prompt Processing}, within the limits of the available network. i
The Prompt Product Database was  stood up and tested.  The  \gls{Science Platform} was deployed and users simulated.
The later of course is designed around kubernetes and almost made for Google Cloud.


We did not demonstrate teh other major (and largest part of \gls{DM}) the \gls{Data Release} Processing. That we are
currently trying with
Amazon Web Services / Elastic Compute Cloud and HTCondor.
See \citeds{DMTN-114} for details.
This is progressing.

We ay conclude from these studies that we have good people in \gls{DM} and
we are able to deploy our systems in various locations especially if kubernetes is avialble.

\input{cost}


\section{Other issues to consider}
We may still wish to keep a copy of data out of the cloud e.g. in Chile and/or NCSA.

We have not considered bulk transfer of data to other partners - this may be far too expensive via google.

\section{Conclusion}

LSST \gls{DM} has been constructing a cloud ready system for many years. We believe commercial cloud is the correct approach but we may be a few years ahead of commercial and federal cost models aligning. We hope that we may be able to partner with google to usher in a new ear of federally funded research in the cloud.

~


