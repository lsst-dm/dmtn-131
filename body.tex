\section{Introduction}
The Large Synoptic Survey Telescope (LSST) will usher in a new era of astronomy, the observing program will observe the southern sky repeatedly over 10 years in 6 bands providing an unprecedented  census of the astrophysical bodies in the universe.  Funded bu NSF and DOE this key observatory is due to go into operations in October 2022.

Producing 20TB a night this is a huge step up in data acquisition from other optical telescopes. At its conception it was considered an ominous data volume requiring highly specialized computing infrastructure. These days, however, industry is handling this sort of data flow without problems.

We have made some experiments to show that we could potentially use commercial cloud providers for at least a large part of the LSST Data Management.  The barrier to acting on this is price and uncertainty on future pricing.

A solution might be to agree a fixed price partnership on the experiment where google agree to \emph{do what is needed} for success at some reasonable and agreed annual fee.

\section{Studies to date}
A simplified sizing model (\citeds{DMTN-072}) was used to run a Google study on \gls{LSST}, a detailed
report may be found in \citeds{DMTN-125}.

We deomstrated some of the major compnents of \gls{LSST} Data Managmenet could work on Google.
Brifely we deployed \gls{Qserv} on Google with reasonable performance (80\% or better of in-house).
We demonstrated Data transfer  could adequate for \gls{Prompt Processing}, within the limits of the available network. i
The Prompt Product Database was  stood up and tested.  The  \gls{Science Platform} was deployed and users simulated.
The later of course is designed around kubernetes and almost made for Google Cloud.


We did not demonstrate teh other major (and largest part of \gls{DM}) the \gls{Data Release} Processing. That we are
currently trying with
Amazon Web Services / Elastic Compute Cloud and HTCondor.
See \citeds{DMTN-114} for details.
This is progressing.

We ay conclude from these studies that we have good people in \gls{DM} and
we are able to deploy our systems in various locations especially if kubernetes is avialble.


\section{Cost}
Currently the cloud computing cost models do  not align well with federal research computing plans. We believe this gap will narrow or disappear in five years or so, however in the meantime LSST and Google may miss an opportunity to move a major research project on to commodity cloud computing.

Using the information from the Google study where we ran some of our real processes, we have come to a price for running the \gls{Science Platform} and storing data on Google. For 15PB of storage and a modest K8S cluster to host the platform the projected 2022 cost is around \$3M of which more than $\frac{2}{3}$ are storage costs.

Though a good price this is not a sustainable price for \gls{LSST}, we can construct petascale storage we would own and use for 5 years for under \$200K a petabyte (the implied \emph{annual} price at google).  We require about 50 Petabytes a year for 10 years. The out year costs look prohibitive on the cloud.

The cost of compute is probably not an issue in comparison - we can use spot/interuptable instance pricing for \gls{DRP}.

Based on the proof of concept some prices were calculated in \tabref{tab:Google}.

\tiny \begin{longtable} { |p{0.22\textwidth}  |r  |r  |r |} 
\caption{Price estimates from google POC \label{tab:Google}}\\ 
\hline 
{Google Compute}&{per month}&{1 year price}&{Price GFLOP} \\ \hline
{POC price GFLOP (n1-highmen-4)}&{\$242.00}&{\$2,904.00}&{\$2.60} \\ \hline
{LIkely (inefficeny included)}&{}&{}&{\$5.34} \\ \hline
{Pessimistic (double that)}&{}&{}&{\$10.69} \\ \hline
{Google Storage}&{GB/ m}&{TB/ month}&{note } \\ \hline
{Optm. }&{0.007}&{\$84.00}& \\ \hline
{Likely (HDD)}&{0.03999975641}&{\$480.00}&{26cent on web} \\ \hline
{Pessimiistic (SSD)}&{0.1699908088}&{\$2,039.89}& \\ \hline
\end{longtable} \normalsize



\section{Conclusion}
